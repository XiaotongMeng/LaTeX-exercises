\documentclass{scrartcl} 
\usepackage[T1]{fontenc} 
\usepackage[utf8]{inputenc} 
\usepackage[ngerman]{babel} 
\usepackage{marvosym} 
\usepackage{lmodern} 
\tolerance=2000 
\setlength{\emergencystretch}{20pt} 
\usepackage{pifont}
\usepackage{pbsi}
\usepackage{yfonts}
\usepackage{array}
\usepackage{parskip}
\usepackage{caption}
\usepackage{tabularx}


\begin{document}
{\bfseries Blluthochdruck(Hypertonie, Hypertonus)}\par\bigskip
Der ideale Blutdruck liegt nach Angaben der Hochdruckliga bei 120/80 mmHg. Von
Bluthochdruck (Hypertonie oder Hypertonus) spricht man, wenn der Druck in den Arterien
krankhaft auf einen systolischen Wert von über 140 mmHg und einen diastolischen
Wert über 90 mmHg gesteigert ist.\\ \\
Die Entscheidung, ob der Blutdruck behandlungsbedürftig ist oder nicht, hängt nicht nur
von der Druckhöhe ab, sondern vom Gesamtrisiko für einen Herzinfarkt oder Schlaganfall.
Eine entscheidende Rolle spielen weitere Krankheiten, beispielsweise das Metabolische
Syndrom oder ein Typ-2-Diabetes. {\itshape Eine grobe Zuordnung kann der Tabelle \ref{table:Tabelle 1} auf
Seite \pageref{table:Tabelle 1} entnommen werden}.\\ \\
Auch der als "hochnormal"bezeichnete Blutdruck zwischen 130–139/85–89 mmHg kann
schon Schäden verursachen. Das gilt vor allem, wenn weitere Gefäßrisikofaktoren vorliegen.
Besteht ein hohes Gesamtrisiko, gehen die Empfehlungen mittlerweile dahin, auch
bei diesen Werten schon mit Medikamenten einzugreifen.

\begin{table}[!htb]
\centering
\begin{tabular}{|>{\itshape}l c c|}\firsthline
\multicolumn{3}{|>{\bfseries}c|}{Blutrucktabelle}\\\hline\hline
 & systolisch{\itshape(mmHg)} & diastolisch{\itshape(mmHg)}\\
 Optimal & < 120 & < 80\\
Normal & 120 – 129 & 80 – 84\\
Hochnormal & 130 – 139 & 85 – 89\\
Leichter Bluthochdruck & 140 – 159 & 90 – 99\\
Mittelschwerer Bluthochdruck & 160 – 179 & 100 – 109\\
Schwerer Bluthochdruck & $\geq$ 180 & $\geq$ 110\\
Isolierter systolischer Bluthochdruck & $\geq$ 140 & < 90\\\lasthline
\end{tabular}
\caption{Eine Übersicht über Blutdruckmesswerte und deren medizinische Einordnung}
\label{table:Tabelle 1}


\end{table}

\end{document}
