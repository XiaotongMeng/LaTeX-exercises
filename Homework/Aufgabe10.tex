\documentclass{scrartcl} 
\usepackage[T1]{fontenc} 
\usepackage[utf8]{inputenc} 
\usepackage[ngerman]{babel} 
\usepackage{marvosym} 
\usepackage{lmodern} 
\tolerance=2000 
\setlength{\emergencystretch}{20pt} 
\usepackage{pifont}
\usepackage{pbsi}
\usepackage{yfonts}
\bibliographystyle{alpha}
\usepackage{makeidx}
\makeindex

\begin{document}

\section*{Energiespartipps für Strom\index{Strom}, Gas und Wasser\index{Wasser}}
Mit diesen Tipps verbrauchen Sie weniger Energie, sparen dadurch Geld und tun zugleich etwas für die Umwelt. Da Energie ein immer knapper werdendes Gut ist, werden die Preise voraussichtlich hoch bleiben. Daher ist es wichtig, sich heute schon Gedanken zum Thema Energiesparen zu machen und neben dem eigenen Geldbeutel auch Klima und Umwelt zu schonen. Bereits mit wenigen einfachen Maßnahmen können Sie Ihren Energieverbrauch senken:\\

\section*{Beim Heizen\index{Heizen|(}}
\begin{itemize}
\item Senken Sie die Raumtemperatur:\index{Temperatur} ein Grad reduziert den Heizenergieverbrauch\index{Heizen} um sechs Prozent.
\item Senken Sie die Vorlauftemperatur\index{Temperatur} des Heizkessels (soweit möglich) ab, um Verteilverluste zu reduzieren.
\item Entlüften Sie ggf. die Heizkörper, nur dann wird die volle Heizleistung erreicht.
\item Stellen Sie die Heizung auf Nachtabsenkung oder drehen Sie abends und bei Abwesenheit die Ventile bzw. die Thermostate herunter.
\item Beheizen\index{Heizen} Sie nicht genutzte Räume nur mit geringeren Temperaturen\index{Temperatur}.
\item Lüften Sie richtig: Mehrmals am Tag Fenster weit öffnen, währenddessen Thermostatventile schließen. Die Öffnungszeit soll ca. 2 min (bei Außentemperatur\index{Temperatur} -10 $^{\circ}$C) bis 15 min (bei Außentemperatur\index{Temperatur} 15 $^{\circ}$C) betragen \cite{griesshammer2008}.
\item Vermeiden Sie „Dauerlüften“ bei gekipptem Fenster.
\item Schließen Sie die Rollläden in der Heizperiode nach Einbruch der Dunkelheit.
\end{itemize}

\section*{Beim Kühlen\index{Kuehlen|(}- und Gefrieren}
\begin{itemize}
\item
Ein Herd direkt neben einem Kühl-\index{Kuehlen|(} oder Gefriergerät erhöht dessen Verbrauch -trennen Sie die Geräte möglichst.
\item
Öffnen Sie die Tür des Kühlschranks nur so kurz wie nötig, denn eindringende warme Luft führt leicht zur Eisbildung und damit zur Erhöhung des Stromverbrauchs\index{Strom}.
\item
Gegarte Speisen, die Sie kühl stellen oder einfrieren möchten, sollten erst vollständig an der Außenluft abkühlen\index{Kuehlen}, bevor sie in den Kühl- \index{Kuehlen|)}oder Gefrierschrank kommen.
\item
Kontrollieren Sie regelmäßig die Temperatureinstellung\index{Temperatur}. Im Kühlschrank sind 7 $^{\circ}$C vollkommen ausreichend \cite{schlumberger2006}. Für Gefriergeräte reicht eine Temperatur\index{Temperatur} von -18 $^{\circ}$C.
\end{itemize}

\section*{Bei der Beleuchtung}
\begin{itemize}
\item Setzen Sie Energiesparlampen ein. Diese verbrauchen bei gleicher Lichtleistung etwa 80 Prozent weniger Strom\index{Strom} als die herkömmlichen Glühlampen und haben eine acht- bis zehnmal so lange Lebensdauer.
\end{itemize}

\section*{Beim Waschen \index{Waschen|(} und Trocknen\index{Trocknen|see{Waschen}}}
\begin{itemize}
\item Kochwäsche wird mit modernen Waschmitteln auch bei 60 $^{\circ}$C statt bei 90 $^{\circ}$C
sauber; für Buntwäsche reichen 30 $^{\circ}$C bis 40 $^{\circ}$C.
\item Nutzen Sie die Füllmenge der Waschmaschine immer optimal aus
\item Verzichten Sie bei normal verschmutzter Wäsche auf den Vorwaschgang.
\item Trocknen\index{Trocknen|see{Waschen}} Sie - wann immer möglich - auf der Leine, denn konventionelle Wäschetrockner haben einen sehr hohen Stromverbrauch\index{Strom}. Steigen Sie ansonsten auf Wärmepumpen- oder Gastrockner um.
\end{itemize}

\section*{Beim Kochen}
\begin{itemize}
\item Verwenden Sie Töpfe mit gut schließendem Deckel.
\item Verwenden Sie Töpfe mit Böden, die die Wärme gut leiten.
\item Garen Sie Gemüse wie z.B. Kartoffeln mit wenig Wasser (schont auch die Vitamine).
\item Schalten Sie das Kochfeld kurz vor Ende der Garzeit ab, um die Restwärme zu nutzen.
\item Heizen \index{Heizen|)}Sie den Backofen nur vor, wenn es unbedingt erforderlich ist.
\end{itemize}

\section*{Im Bad}
\begin{itemize}
\item Nehmen Sie eine (kurze) Dusche statt eines Vollbades.
\item Verwenden Sie Sparduschköpfe; der Wasser- und Energieverbrauch sinkt pro Duschvorgang um bis zu 50 Prozent.
\item Nehmen Sie, wann immer möglich, kaltes statt warmes Wasser (z.B. beim).
\item Stellen Sie warmes Wasser beim Einseifen, Rasieren oder Zähneputzen ab.
\item Verzichten Sie ganz auf Zirkulationsleitungen oder setzen Sie eine zeitgesteuerte Schaltuhr ein.
\end{itemize}

\section*{Beim Kauf von Haushaltsgeräten}
\begin{itemize}
\item Achten Sie bei Neuanschaffungen besonders auf die Verbrauchsdaten bzw. die Energieeffizienz. Ein Aufkleber auf dem jeweiligen Gerät gibt die Energieeffizienzklasse an und benennt die Verbrauchsdaten. Kaufen Sie keine Geräte, die nicht mindestens der Energieeffizienzklasse A angehören.
\end{itemize}

\section*{Stand-by}
\begin{itemize}
\item Vermeiden Sie „Stand-by-Verluste“, z.B. durch den Einsatz einer schaltbaren Steckerleiste oder eines Vorschaltgerätes.\index{Wasser|)}
\item Berücksichtigen Sie beim Kauf von Neugeräten die Höhe der „Stand-by-Verluste“. Bevorzugen Sie Geräte, die komplett abgeschaltet werden können.
\item Schalten Sie PC und umgebende Geräte aus, wenn Sie diese nicht nutzen.
\end{itemize}

\printindex
\bibliography{bibliographie}


\end{document}