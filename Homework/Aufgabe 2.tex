\documentclass{scrartcl}
\usepackage[T1]{fontenc}
\usepackage[utf8]{inputenc}
\usepackage[ngerman]{babel}
\usepackage{marvosym}
\DeclareUnicodeCharacter{20AC}{\EUR}


\begin{document}


\section*{Zusammenfassung}
Es folgt ein formatierter Text mit automatischen Inhaltverzeichnis.



\tableofcontents

\section{Beschreibung}
\subsection{Aufgabe}
Mit der Musterpäämbel soll ein neues Doukument  erstellt.Diese Dokument soll mit einer nicht-nummerierten Zusammenfassung beginnen.
\subsection{Deutscher Text}
Es muss darauf geachtet werden,dass in der Präambel  alle Einstellung für Deutsche Text eingeführt werden.
\section{Inhaltsangabe}
Das Dokumen soll eine automatische INhaltsangabe erthalten,die nach der Zusammenfassung auszugeben ist.
\section{Unterstrukturen}
Hier ist die oberste Gliederungsebene für den Artikel.
\subsection{Nummerierung}
Die Nummerierung erfolgt automatisch und hierarchisch.
\subsubsection{Änderung von Nummern}
Es sind keine mannuellen Änderung der Nummerierung notwendig.Neu eingeschobene Abschinitte werden automatisch korrekt nummeriert.
\paragraph{Paragraph} Paragraphen werden zwar intern nummeriert,aber in der Standareinstellung werden diese Nummern nicht ausgegeben.Auderdem werden die Paragraphen in der Standardeinstellung auch nischt in das Inhaltverzeichnis geschrieben.
\subsubsection{Referenzen aud Nummern} 
\label{mylabel}
Im Kurs wird gezeigt,wie man auf logische Gliederungsnummern durch Querbezüge
verweisen kann (siehe unten).
\section{Formatisierung}
\subsection[Lange Überschrift]{Auch sehr lange Überschriften können verwendet werden, wobei hier eine Kurzfassung ins Inhaltsverzeichnis geschrieben werden sollte}
Blicke in das Inhaltsverzeichnis.
\subsection{Fettdruck}
Größe und Fettdruck von Überschriften wird automatisch eingestellt.
\subsection{Abstände}
Abstände werden ohne weitere Einstellungen automatisch gesetzt.
\section{Referenzieren}
Es kann auch auf ein {\itshape label} verwiesen werden. In diesem Fall ist das {\itshape label} in Abschnitt [\ref{mylabel}] auf Seite [\pageref{mylabel}].

\end{document}