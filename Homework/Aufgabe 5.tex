\documentclass{scrartcl} 
\usepackage[T1]{fontenc} 
\usepackage[utf8]{inputenc} 
\usepackage[ngerman]{babel} 
\usepackage{marvosym} 
\usepackage{lmodern} 
\tolerance=2000 
\setlength{\emergencystretch}{20pt} %verhindert das herrausragen von Wörtern übers Zeilenende
\usepackage{pifont}
\usepackage{pbsi}
\usepackage{yfonts}
\usepackage{parskip}

\begin{document}
{\centering {\large{\bfseries Rezept für Thai - Red - Curry}}\\}\par\bigskip
{\itshape {\bfseries Zutaten:}}\\ \\
\begin{tabbing}

\hspace{1cm}\=\hspace{1cm}\=\kill
\> 300\'g\> Fleisch oder Fisch nach Wahl\\
\> 1\'EL \> Currypaste, rote\\
\> 200\'ml \> Wasser\\
\> 400\'ml \> Kokosmilch\\
\> 800\'g\> Gemüse nach Wahl\\
\> 2 \'EL \> Fischsauce\\
\> 2 \'EL \> Sojasauce, helle\\
\> 2 \'St.\> Peperoni, rot oder grün, schräg geschnitten\\
\> 2 \'St.\> Chilischote(n), kleine scharfe (nach Belieben)\\
\> 6 \'St.\ Thai-Basilikum Blätter\\
\> 2 \'EL \> Rapskernöl oder Erdnussöl\\ \\
\end{tabbing}

{\itshape {\bfseries Zubereitung:}}\\ \\
Dieses Grundrezept kann je nach Lust und Laune durch Wahl des Fleisches (z.B. Hühnerbrust,
Putenbrust, Rinderlende oder Schweinefilet) oder mit Fischfilet oder Garnelen
und mit verschiedenem Gemüse (z. B. Bambussprossen in Streifen, Sojabohnenkeimlinge,
Karotten, Babymais, Zuckerschoten, Thai-Auberginen) variiert werden.\\ \\
Die Currypaste im heißen Öl sautieren, kurz mit etwas Wasser ablöschen, nach und nach
die Kokosmilch dazugeben und gut verrühren. Das vorgesehene Fleisch (oder Fisch oder
Garnelen) in mundgerechte Stücke schneiden, dazugeben und ca. 5 Minuten köcheln
lassen, bis es gar ist. Garnelen brauchen nur sehr kurze Zeit!\\ \\
Das in Streifen geschnittene Gemüse der Wahl dazugeben und alles wieder zum Kochen
bringen. Alles sollte bissfest bleiben und die Farbe behalten.\\ \\
Mit der Fischsauce, der hellen Sojasauce und dem Palmzucker abschmecken. ThaiBasilikum-Blätter
und Peperoni hinzufügen, eine Minute weiterkochen. Nach Belieben
und Schärfe-Empfinden die kleingeschnittenen Chili-Röllchen einstreuen.\\ \\
Das Gericht heiß mit Reis (Basmati, Jasminreis oder thailändischer Duftreis) servieren.




\end{document}