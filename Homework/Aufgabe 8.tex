\documentclass{scrartcl} 
\usepackage[T1]{fontenc} 
\usepackage[utf8]{inputenc} 
\usepackage[ngerman]{babel} 
\usepackage{marvosym} 
\usepackage{lmodern} 
\tolerance=2000 
\setlength{\emergencystretch}{20pt} 
\usepackage{pifont}
\usepackage{pbsi}
\usepackage{calc}
\usepackage{yfonts}
\usepackage{fancyhdr}
\pagestyle{fancy}
\fancyhf{}
\usepackage{multicol}
\usepackage{parskip}
\setlength{\parindent}{0pt}
\begin{document}
\renewcommand{\headrulewidth}{1pt}
\renewcommand{\footrulewidth}{1pt}
\fancyhf[FL]{\bfseries\thepage}
\fancyhead[HL]{\sffamily{\bfseries Der Kakapo {\itshape(Strigops habroptilus)}}}

\section*{Kakapo}
Der {\bfseries Kakapo} {\itshape (Strigops habroptilus)} ist ein Papagei, der in Neuseeland beheimatet ist. Er ist die einzige Art der Unterfamilie Strigopinae (Eulenpapageien). Der nachtaktive Vogel ist im Wesentlichen ein Pflanzenfresser. Er ist der einzige bekannte flugunfähige Papagei. Der Kakapo ist akut vom Aussterben bedroht.

\section*{Aussehen}

\begin{multicols}{2}
Alle heute bekannten Kakapos zeichnen sich durch ein moosgrünes Gefieder aus, das am Rücken schwarze Streifen aufweist. Der Unterkörper, der Nacken und das Gesicht sind eher grüngelblich befiedert, wobei die Färbung individuell stark variiert. Von Vogelbälgen in wissenschaftlichen Sammlungen weiß man jedoch, dass es auch Exemplare gegeben hat, die völlig gelblich befiedert waren. Das Gefieder ist ungewöhnlich weich, darauf bezieht sich die Artbezeichnung {\itshape habroptilus}.\\
Kakapos haben einen sogenannten Gesichtsschleier; das heißt, das Gesicht ist von feinen Federn umgeben, wie es für Eulen typisch ist. Hierdurch erklärt sich der lateinische Artname {\itshape Strigops}. Die europäischen Einwanderer auf Neuseeland nannten daher den Kakapo auch Eulenpapagei. Den Schnabel umgeben feine Schnabelborsten, mit denen nachts Hindernisse geortet werden. Die Enden der Schwanzfedern sind meistens zerschlissen, da sie ständig am Boden entlanggezogen werden. \\
Kakapos sind sehr große Papageien; ausgewachsene Männchen messen bis zu 60 Zentimeter und wiegen zwischen drei und vier Kilogramm. Die Flügel sind relativ klein, und es fehlt ihnen das verstärkte Brustbein, an dem die kräftige Flugmuskulatur anderer Vögel ansetzt. Sie gebrauchen ihre Flügel nur zum Balancieren und um ihren Fall abzubremsen, wenn sie von Bäumen herabspringen. Anders als andere Landvögel können Kakapos große Mengen Depotfett speichern. Der Schnabel des Kakapos ist geeignet, Nahrung sehr fein zu zerkleinern. Kakapos haben kleine Kröpfe. Die Füße sind groß und schuppig und haben wie bei allen Papageien zwei nach vorne und zwei nach hinten gerichtete Zehen. Ihre ausgeprägten Krallen sind an das Klettern angepasst (Adaption). 
Eines der ungewöhnlichsten Charakteristika der Kakapos ist ihr starker, aber angenehmer Geruch, der dem Geruch von Blumen und Honig oder Bienenwachs ähnelt.
\end{multicols}

\section*{Verbreitung und Lebensraum}
\begin{multicols}{2}
Die Kakapos besiedelten früher beide neuseeländische Hauptinseln. Der Lebensraum der Kakapos umfasste unterschiedliche Habitate, darunter alpine Heiden, Buschland wie auch küstennahe Bereiche. Sie bewohnten außerdem eine Vielzahl unterschiedlicher Waldformen, in denen Steineibengewächse ({\itshape Podocarpaceae}) (vor allem Rimu ({\itshape Dacrydium cupressinum})), Scheinbuchen, Tawa ({\itshape Beilschmiedia tawa}) oder Eisenhölzer ({\itshape Metrosideros sp.}) dominierten. Bevorzugt wurden dabei Waldrandzonen oder Waldbereiche in jungen Sukzessionsstadien, da diese ihnen eine größere Vielfalt an Nahrung boten. In den Fjordgebieten Neuseelands nannte man die Bereiche, in denen nach Lawinenabgängen oder Erdrutschen junge Wälder mit einem dichten, fruchttragenden Strauchwerk aufwuchsen, Kakapo-Gärten.\\ 
Alle Kakapos, von deren Existenz man weiß, sind heute (2008) aus Schutzgründen überwiegend auf zwei kleine Inseln umgesiedelt worden: Anchor Island (Pukenui), (sie liegt im Dusky Sound einem Teil des Fiordland-Nationak) und Codfish Island (Whenua Hou), welche vor der Westküste von Stewart Island liegt.
\end{multicols}

\section*{Verhalten und Nahrung}
\begin{multicols}{2}
Kakapos sind nachtaktiv. Tagsüber ruhen sie versteckt in Bäumen oder am Erdboden; nachts streifen sie durch ihr Revier. Sie können nicht fliegen, sind jedoch exzellente Kletterer, die bis in die Kronen der höchsten Bäume klettern. Man hat beobachtet, wie sie von diesen Höhen fallschirmähnlich herabgleiten, indem sie ihre Flügel spreizen und dadurch ihren Fall abbremsen. Kakapos sind ausgezeichnete Läufer; während einer Nacht können sie mehrere Kilometer zurücklegen und dabei hunderte von Höhenmetern überwinden. Sie können auch mit einem beachtlichen Tempo rennen, halten eine hohe Geschwindigkeit aber nicht über eine längere Distanz.\par\medskip
Kakapos ernähren sich überwiegend von einer großen Zahl von Pflanzen, Samen, Früchten, Pollen und sogar vom Baumsaft von Bäumen. Mit besonderer Vorliebe fressen sie die Früchte des Rimu-Baums und ernähren sich ausschließlich davon, wenn diese Früchte reichlich vorhanden sind. Blätter werden häufig mit einem Fuß festgehalten, um dann mit dem Schnabel die nahrhaften Teile abzustreifen, so dass die hartfaserigen Blattbestandteile überbleiben. Die Reste solcher Blätter sind ein eindeutiges Kennzeichen der Anwesenheit von Kakapos. Man hat darüber hinaus auch beobachtet, dass Kakapos Insekten und andere wirbellose Tiere fressen.\\ 
Kakapos sind von Natur aus sehr neugierig und reagieren mitunter sogar interessiert auf gelegentlich anwesende Menschen. Wie andere Papageien auch verfügen Kakapos über eine große Bandbreite unterschiedlicher Rufe, die verschiedene Funktionen haben. Zusätzlich zu den booms und chings ihrer Balzrufe, geben sie beispielsweise mit einem skraark ihre Anwesenheit anderen Vögeln bekannt.
\end{multicols}

\section*{Verhalten gegenüber Räubern}
\begin{multicols}{2}
Kakapos haben wie viele flugunfähige Inselformen kein Feindverhalten gegen Bodenprädatoren, da Neuseeland ursprünglich frei von solchen Prädatoren war. Wenn Kakapos sich bedroht fühlen, erstarren sie und verlassen sich auf ihre Tarnung. Dieses Verhalten ist ein geeigneter Schutz gegenüber Adlern, die früher ihre einzigen Feinde waren, es schützt sie jedoch nicht vor den durch Menschen eingeführten Raubtieren, die vor allem ihren Geruchssinn bei der Nahrungssuche nutzen.
\end{multicols}

\section*{Systematik}
\begin{multicols}{2}
Der Kakapo ist der einzige Vertreter der Gattung Strigops, die meist als Unterfamilie zu den Eigentlichen Papageien gestellt wird. Über die genaue phylogenetische Position ist nichts bekannt, meistens werden sie als ursprünglichste Form allen anderen Echten Papageien gegenüber gestellt und stellen somit die Schwestergruppe all dieser Gruppen dar.
\end{multicols}

\end{document}