\documentclass{scrartcl} 
\usepackage[T1]{fontenc} 
\usepackage[utf8]{inputenc} 
\usepackage[ngerman]{babel} 
\usepackage{marvosym} 
\usepackage{lmodern} 
\tolerance=2000 
\setlength{\emergencystretch}{20pt} 
\usepackage{pifont}
\usepackage{pbsi}
\usepackage{yfonts}
\usepackage{parskip}
\usepackage{caption}
\usepackage[bottom]{footmisc}
\usepackage{calc}
\usepackage{amsmath}
\usepackage{empheq}
\usepackage{amssymb}
\usepackage{mathrsfs}
\usepackage{theorem}

\begin{document}

\section*{Tesla}
{\bfseries Tesla (T)} ist eine abgeleitete SI-Einheit für die magnetische Flussdichte. Die Einheit wurde im Jahre 1960 auf der Conf'{e}rence G\'{e}n\'{e}rale des Poids et Mesures (CGPM) in Paris nach Nikola Tesla benannt.\\
\section*{In SI-Einheiten}
\begin{equation}
1T = 1\frac{kg}{As^2}=1\frac{Vs}{m^2} 
\end{equation}
\section*{In elektrostatischen CGS-Einheiten (CGS-ESU)}
\begin{equation}
1T=\frac{1}{3}\cdot10^{-6}\sqrt{\frac{g}{cm^3}}
\end{equation}
\section*{In elektromagnetischen CGS-Einheiten (CGS-EMU)}
\begin{equation}
1T=10^4\frac{\sqrt{g}}{\sqrt{cm}s}
\end{equation}

\end{document}