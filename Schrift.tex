\documentclass{scrartcl} 
\usepackage[T1]{fontenc} 
\usepackage[utf8]{inputenc} 
\usepackage[ngerman]{babel} 
\usepackage{marvosym} 
%\usepackage{lmodern} 
\usepackage{parskip}
\tolerance=2000 
\setlength{\emergencystretch}{20pt}
\usepackage{yfonts}
\usepackage{enumerate}

\begin{document}


\begin{sffamily} %wichtig nicht vergessen!!!!!!%
Eine \textit{Kombination aus \textbf{Schriftfamilie}}, {\slshape {\ttfamily Schriftserie} und}{\scshape Schriftform}.
\end{sffamily}  %wichtig nicht vergessen%


%%%%%%%%%%%%%%%%%%%%%%%%%%%%%%%%%%%%%%%%%%%%%%%%
Dieses Dokument benutzt die Dokumentklasse\glq article\grq{} und hat eine Schriftgröße von 11 pt.Hier beginne wir mit {\Huge\glqq Riesiger \grqq}.Beachtet also,dass die Befehle \flqq case sensitive\frqq{} sind.
{\large\glqq\textbackslash large \grqq},
{\Large\glqq\textbackslash Large \grqq} und
{\LARGE\glqq\textbackslash lARGE}
sind nicht die selben Befehle.
Ander......auch {tiny\glqq winzige \grqq} Schriftgrößen einstellen.
%%%%%%%%%%%%%%%%%%%%%%%%%%%%%%%%%%%%%%%%%%%%%%%%%



\begin{verse}
\textsc{Von Kapitälchen:} OHA! Wer sind sie Denn? Sie haben blabla .\\
\textbf{Schwabacher:} Mein Name ist Prinzessin  blabla.\\
\textsc{Von Käpitälchen:}:OK blabla kein Bock mehr abzuschreiben.
\end{verse}


%%%%%%%%%%%%%%%%%%%%%%%%%%%%%%%%%%%%%%%%%%%%%%%%%%%%%

\begin{itemize}
\item Erster Aufzählungspunkt auf der ersten Stufen
 \begin{enumerate}
 \item Erster Unterpunkt auf der zweiten Ebene
  \begin{description}
  \item[Einschub:] Dritte  Ebene
  
      \begin{enumerate}
      \item Vierte Ebene
       \begin{itemize}
         \item Fünfte Ebene
       \end{itemize}
      \item Zweite Punkt aud der vierte Ebene
    \end{enumerate}
  
   \end{description}

  \end{enumerate}

\item Wieder eine Eben

\end{itemize}



%%%%%%%%%%%%%%%%%%%%%%%%%%%%%%%%%%%%%%%%%%%%%%%%%%


\begin{tabbing}
\textbf{Es folgt die erste Tabbing Umgebung:} \= \\
\> ,um Problem zu vermeiden, \\
soll der erste Tabstopp erst nach der längsten Zeilen erfolgen.

\end{tabbing}


%%%%%%%%%%%%%%%%%%%%%%%%%%%%%%%%%%%%%%%%%%%%


\begin{tabbing}
Es folgt die erste Tabbing Umgebung:\= \+\\
ersten Zeile\\
zweite Zeile\\ 
\end{tabbing}

%%%%%%%%%%%%%%%%%%%%%%%%%%%%%%%%%%%%%%%%%%%%


\begin{tabbing}
\textbf{Es folgt die zweiten Tabbing Umgebung:}  \\ \\
Erster Tabstopp.\= Die nächste Zeile soll beim ersten Tabstopp beginnen.\+ \\ 
 Hier ein weiterer Tabstopp. \= Die nächste Zeile beginnt bei Null.\- \\
Die folgende Zeile beginnt beim zweite Tabstopp.\+\+ \\ 
 Perfekt!
\end{tabbing}

%%%%%%%%%%%%%%%%%%%%%%%%%%%%%%%%%%%%%%%%%%%%%

\begin{tabbing}
\textbf{Es folgt die dritte Tabbing Umgebung:}  \\ \\
Fangen wer mit eimem Tabstopp an. \= Und setzen hier noch einen. \= \\    
\>\>Rechtsbündig zum zweiten Tapstopp. \' \\
\> Linksbündig zum ersten.\\
\>\> Das was.
\end{tabbing}









\end{document}
