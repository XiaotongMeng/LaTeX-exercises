\documentclass{scrartcl} 
\usepackage[T1]{fontenc} 
\usepackage[utf8]{inputenc} 
\usepackage[ngerman]{babel} 
\usepackage{marvosym} 
\usepackage{lmodern} 
\usepackage{parskip}
\tolerance=2000 
\setlength{\emergencystretch}{20pt}
\usepackage{array} %!!!!!%
\usepackage[bottom]{footmisc}


\begin{document}
\listoftables

\begin{table}[!b]
\centering
\begin{tabular}{|c|c|c|}\firsthline
\multicolumn{3}{|c|}{Ersten Tabelle }\\\hline
Die&erste&Tabelle \\
ist&ganz&einfach \\\lasthline
\end{tabular}
\caption{Grundsätzliche Formatisierung}
\end{table}


\begin{table}[!b]
\centering
\begin{tabular}
{|>{\itshape}p{1,5cm}|>{\bfseries}m{1,5cm}|b{1,5cm}|}\firsthline
\multicolumn{3}{|c|}{Zweite Tabelle }\\\hline
Hier wird der Text vertikal ausgerichtet & 
Aber auch der Textformatiert & 
Man beachte auch die Spaltenbreite {1,5cm} \\\lasthline
\end{tabular}
\caption{Weitere Ausrichting}
\end{table}


\begin{table}[!b]
\centering
\begin{tabular}{|>{\itshape}l||>{\bfseries}c|r|}\firsthline
\multicolumn{3}{|c|}{Dritte Tabelle }\\\hline\hline
Hierarbeiten & wir mit & unterschiedlichen\\\cline{1-2}
vertikal und & horizonslr & Linien \\\cline{2-3} 
und & horizonalen & Ausrichtungen \\\lasthline
\end{tabular}
\caption{Linien und horizonale Ausrichtung}
\end{table}








\end{document}
